%---- Required Packages and Functions ----

\documentclass[a4paper,11pt]{article}
\usepackage{latexsym}
\usepackage{xcolor}
\usepackage{float}
\usepackage{ragged2e}
\usepackage[empty]{fullpage}
\usepackage{wrapfig}
\usepackage{lipsum}
\usepackage{tabularx}
\usepackage{titlesec}
\usepackage{geometry}
\usepackage{marvosym}
\usepackage{verbatim}
\usepackage{enumitem}
\usepackage[hidelinks]{hyperref}
\usepackage{fancyhdr}
\usepackage{fontawesome5}
\usepackage{multicol}
\usepackage{graphicx}
\usepackage{cfr-lm}
% \usepackage[T1]{fontenc}
\setlength{\footskip}{4.08003pt} 
\pagestyle{fancy}
\fancyhf{} % clear all header and footer fields
\fancyfoot{}
\renewcommand{\headrulewidth}{0pt}
\renewcommand{\footrulewidth}{0pt}
\geometry{left=1.0cm, top=1cm, right=1cm, bottom=1cm}
% Adjust margins
%\addtolength{\oddsidemargin}{-0.5in}
%\addtolength{\evensidemargin}{-0.5in}
%\addtolength{\textwidth}{1in}
\usepackage[most]{tcolorbox}

\usepackage[T2A]{fontenc}

%Hyphenation rules
%--------------------------------------
\usepackage{hyphenat}
%--------------------------------------
\usepackage[english, russian]{babel}

\tcbset{
	frame code={}
	center title,
	left=0pt,
	right=0pt,
	top=0pt,
	bottom=0pt,
	colback=gray!20,
	colframe=white,
	width=\dimexpr\textwidth\relax,
	enlarge left by=-2mm,
	boxsep=4pt,
	arc=0pt,outer arc=0pt,
}

\urlstyle{same}

\raggedright
\setlength{\footskip}{4.08003pt}

% Sections formatting
\titleformat{\section}{
  \vspace{-4pt}\scshape\raggedright\large
}{}{0em}{}[\color{black}\titlerule \vspace{-7pt}]

%-------------------------
% Custom commands
\newcommand{\resumeItem}[2]{
  \item{
    \textbf{#1}{\hspace{0.5mm}#2 \vspace{-0.5mm}}
  }
}

\newcommand{\resumePOR}[3]{
\vspace{0.5mm}\item
    \begin{tabular*}{0.97\textwidth}[t]{l@{\extracolsep{\fill}}r}
        \textbf{#1}\hspace{0.3mm}#2 & \textit{\small{#3}} 
    \end{tabular*}
    \vspace{-2mm}
}

\newcommand{\resumeSubheading}[4]{
\vspace{0.5mm}\item
    \begin{tabular*}{0.98\textwidth}[t]{l@{\extracolsep{\fill}}r}
        \textbf{#1} & \textit{\footnotesize{#4}} \\
        \textit{\footnotesize{#3}} &  \footnotesize{#2}\\
    \end{tabular*}
    \vspace{-2.4mm}
}

\newcommand{\resumeProject}[4]{
\vspace{0.5mm}\item
    \begin{tabular*}{0.98\textwidth}[t]{l@{\extracolsep{\fill}}r}
        \textbf{#1} & \textit{\footnotesize{#3}} \\
        \footnotesize{\textit{#2}} & \footnotesize{#4}
    \end{tabular*}
    \vspace{-2.4mm}
}

\newcommand{\resumeSubItem}[2]{\resumeItem{#1}{#2}\vspace{-4pt}}

% \renewcommand{\labelitemii}{$\circ$}
\renewcommand{\labelitemi}{$\vcenter{\hbox{\tiny$\bullet$}}$}

\newcommand{\resumeSubHeadingListStart}{\begin{itemize}[leftmargin=*,labelsep=0mm]}
\newcommand{\resumeHeadingSkillStart}{\begin{itemize}[leftmargin=*,itemsep=1.7mm, rightmargin=2ex]}
\newcommand{\resumeItemListStart}{\begin{justify}\begin{itemize}[leftmargin=3ex, rightmargin=2ex, noitemsep,labelsep=1.2mm,itemsep=0mm]\small}

\newcommand{\resumeSubHeadingListEnd}{\end{itemize}\vspace{2mm}}
\newcommand{\resumeHeadingSkillEnd}{\end{itemize}\vspace{-2mm}}
\newcommand{\resumeItemListEnd}{\end{itemize}\end{justify}\vspace{-2mm}}
\newcommand{\cvsection}[1]{%
\vspace{2mm}
\begin{tcolorbox}
    \textbf{\large #1}
\end{tcolorbox}
    \vspace{-4mm}
}

\newcolumntype{L}{>{\raggedright\arraybackslash}X}%
\newcolumntype{R}{>{\raggedleft\arraybackslash}X}%
\newcolumntype{C}{>{\centering\arraybackslash}X}%
%---- End of Packages and Functions ------

%-------------------------------------------
%%%%%%  CV STARTS HERE  %%%%%%%%%%%
%%%%%% DEFINE ELEMENTS HERE %%%%%%%
\newcommand{\name}{Константин Рогов} % Your Name
\newcommand{\phone}{929-916-91-01}
\newcommand{\emaila}{rogovkostyas@yandex.ru}



\begin{document}
\fontfamily{cmr}\selectfont
%----------HEADING-----------------


\parbox{\dimexpr\linewidth-0.3cm\relax}{
\begin{tabularx}{\linewidth}{L r} \\
  \textbf{\Large \name} \\ 
  {\raisebox{0.0\height}{\footnotesize \faPhone}\ +7-\phone} & \href{https://github.com/Rogov-KS}{\raisebox{0.0\height}{\footnotesize \faGithub}\ {GitHub Profile}}\\
  \href{mailto:\emaila}{\raisebox{0.0\height}{\footnotesize 
 \faEnvelope}\ {\emaila}}&\href{https://t.me/RogovKostya}{\raisebox{0.0\height}{\footnotesize \faTelegram}\ {Telegram Account}}
\end{tabularx}
}





%-----------EXPERIENCE-----------------
\section{\textbf{Experience}}
  \resumeSubHeadingListStart

    \resumeSubheading
      {Яндекс}{}
      {Стажёр-Аналитик в команде Алисы}{июль - сентябрь 2023 года}
      \vspace{-2.0mm}
      \resumeItemListStart
    \item {Работа с толокерами: подготовка заданий для них, и запуск\остановка разметки}
    \item {Оптимизация yql-запросов}
    \item {Подготовка и настрока дашборда, включая всю необходимую предобработку данных}
    \item {Общение с разработчиками по поводу возможных фичей}
    
    \resumeItemListEnd
    
    \vspace{-3.0mm}
    
      
  \resumeSubHeadingListEnd
\vspace{-8.5mm}



%-----------EDUCATION-----------
\section{\textbf{Education}}
  \resumeSubHeadingListStart
    \resumeSubheading
      {МФТИ - Бакалавриат прикладная математика и информатика}{Средний балл: 8.47}
      {кафедра - Анализ данных (Тинькофф)}{}
    \resumeSubheading
      {Изучаемые курсы}{}
      {АБ - тесты, Мат Стат, ML, NLP, Алгоритмы и структуры данных, АКОС, Многопоточная синхронизация и др.}{}
  \resumeSubheading
      {Лицей 1511 при МИФИ}{}
      {Старшая школа}{года обучения: 20219-2021}
  \resumeSubHeadingListEnd
\vspace{-5.5mm}
%


%-----------PROJECTS-----------------
\section{\textbf{Personal Projects}}
\resumeSubHeadingListStart

    \resumeProject
      {Мультимодальный поиск с помощью CLIP} %Project Name
      {Реализован backend+frontend сервер для мультимодального поиска} %Project Name, Location Name
      {весна 2024, \href{https://github.com/Rogov-KS/MultimodalRetrieval}{github link}} %Event Dates

      \resumeItemListStart
        \item {Мультимодальная модель - CLIP от OpenAI}
        \item {Векторный индекс - FAISS}
        \item {Фреймворк для backend сервера - FastAPI}
        \item {Фреймворк для frontend сервера - ReactJS}
    \resumeItemListEnd
    \vspace{-2.0mm}
    
    \resumeProject
      {Аналог популярных классов из библиотеки std для C++} %Project Name
      {Реализован аналог основных функциональностей классов std::deque, std::list и умных указателей из std} %Project Name, Location Name
      {\href{https://github.com/Rogov-KS/C_plus_plus_tasks}{github link}} %Event Dates

      \resumeItemListStart
    \resumeItemListEnd
    \vspace{-2.0mm}

    \resumeProject
      {Wargame на питоне} %Project Name
      {} %Project Name, Location Name
      {\href{https://github.com/Rogov-KS/TP_Project}{github link}} %Event Dates

    \resumeProject
      {Telegram bot на питоне} %Project Name
      {} %Project Name, Location Name
      {\href{https://github.com/Rogov-KS/Telegram_first_bot}{github link}} %Event Dates

    \vspace{-2.0mm}
    
      
  \resumeSubHeadingListEnd
\vspace{-5.5mm}



%-----------Technical skills-----------------
\section{\textbf{Technical Skills and Interests}}
 \begin{itemize}[leftmargin=0.1in, label={}]
    \small{\item{
     \textbf{Языки программирования:}{\ Python, C++, SQL} \\
     \textbf{Python Libraries:}{\ PyTorch, SciPy, Numpy, Pandas, Seaborn, Matplotlib, Plotly, Prophet} \\
    \textbf{Другие hard skills:}{\ Git, Docker, Bash} \\
     \textbf{Soft Skills:}{\ С одноклассниками из лицея проводил посвященные в лицеисты новых 10ки классников} \\
     % \textbf{Field of Interest}{: } \\
     \textbf{Hobbies:}{\ Спорт (тренажерный зал + бег + игры с мячом)} \\
     \textbf{Языки:}{\ Русский, Английский(intermediate)} \\
    }}
 
 \end{itemize}
 \vspace{-16pt}



% %-----------Positions of Responsibility-----------------
% \section{\textbf{Positions of Responsibility}}
% \vspace{-0.4mm}
% \resumeSubHeadingListStart
% \resumePOR{Position, } % Position
%     {Club or Event} %Club,Event
%     {Position tenure} %Tenure Period
%     -work description in 1 line
% \resumePOR{Position, } % Position
%     {Club or Event} %Club,Event
%     {Position tenure} %Tenure Period
% -work description in 1 line
    
% \resumeSubHeadingListEnd
% \vspace{-5mm}




%-----------Achievements-----------------
\section{\textbf{Achievements}}
\vspace{-0.4mm}
\resumeSubHeadingListStart
\resumePOR{ВСОШ по физике} % Award
    {} % Event
    {Участник, 2021} %Event Year
    
\resumePOR{МОШ по физике} % Award
    {} % Event
    {Победитель, 2021} %Event Year
    
\resumePOR{ПВГ по физике} % Award
    {} % Event
    {Победитель, 2021} %Event Year
    
\resumePOR{Олимпиада ФИЗТЕХ по математике} % Award
    {} % Event
    {Победитель, 2021} %Event Year
\resumeSubHeadingListEnd
\vspace{-5mm}



%-------------------------------------------
\end{document}
